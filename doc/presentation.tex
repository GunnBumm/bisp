\documentclass[t]{beamer}

\usepackage[english]{babel}
\usepackage[utf8]{inputenc}
\usepackage{amsthm}
\usepackage{amsmath}
\usepackage{amsfonts}
\usepackage{color}
\usepackage{graphicx}
\usepackage{graphics}

\newcommand{\textdesc}[1]{\textit{\textbf{#1}}}
\newcommand{\descitem}[1]{\item \textdesc{#1}}
\newcommand{\includeSVG}[1]{
  \includegraphics[scale=1.0]{./figs/#1.pdf}
}
\newcommand{\includeBitmap}[2]{
  \includegraphics[width=#2]{./figs/#1}
}

% remove footer
\setbeamertemplate{footline}[page number]{} % gets rid of bottom navigation bars
\setbeamertemplate{navigation symbols}{} % gets rid of navigation symbols
\usecolortheme{default}
%\useoutertheme{default}
\definecolor{structurecolor}{RGB}{0,0,191}
\setbeamercolor{structure}{fg=structurecolor}
\setbeamertemplate{footline}{}

% PDF settings
\hypersetup{
  pdftitle={Building Information Service Platform, a Discrete Event Simulator for Building Modeling},
  pdfauthor={Aslak Johansen,Matias Bjørling,Javier González González},
  pdfsubject={Long-term environmental monitoring},
  pdfkeywords={Building Simulator}
}

\title{Building Information Service Platform\\\scalebox{0.85}{a Discrete Event Simulator for Building Modeling}}
\author{Aslak Johansen \url{aslj@itu.dk}\\Matias Bjørling \url{mabj@itu.dk}\\Javier González González \url{jgon@itu.dk}}
\logo{
%  \scalebox{1.0}{\includeBitmap{ITU_logo_bart_SH.jpg}{10mm}}
}
\date{February 5, 2013}

\begin{document}

%%%%%%%%%%%%%%%%%%%%%%%%%%%%%%%%%%%%%%%%%%%%%%%%%%%%%%%%%%%%%%%
%%%%%%%%%%%%%%%%%%%%%%%%%%%%%%%%%%%%%%%%%%%%%%%%%%%%%%%%% Intro
\frame{\titlepage}

%%%%%%%%%%%%%%%%%%%%%%%%%%%%%%%%%%%%%%%%%%%%%%%%%%%%%%%%%%%%%%%
%%%%%%%%%%%%%%%%%%%%%%%%%%%%%%%%%%%%%%%%%%%%%%%%%%%%%%% Context
\begin{frame}
  \frametitle{Context}
  
  \begin{itemize}
    \item Climate change is a real problem
    \item We need to lower our carbon footprint in order to limit the damage
    \item Goals have been set for decreasing the use of fossil fuels
    \item Buildings make up a significant part of the consumption so they are a good place to start
    \item The first step in acting is knowing
  \end{itemize}
  
  Energy Futures: \url{http://energyfutures.itu.dk}
  
\end{frame}

%%%%%%%%%%%%%%%%%%%%%%%%%%%%%%%%%%%%%%%%%%%%%%%%%%%%%%%%%%%%%%%
%%%%%%%%%%%%%%%%%%%%%%%%%%%%%%%%%%%%%%%%%%%%%%%%%%%%%%% Problem
\begin{frame}
  \frametitle{Problem}
  
  \begin{itemize}
    \item Testing on a real building has implications
      \begin{itemize}
        \item A bug in the developed logic can bring down the building
        \item People get mad when you bring down the building
      \end{itemize}
    \item Alternative: Use a simulator!
    \item Testing on a simulator has implications
      \begin{itemize}
        \item The modeled world is an approximation
        \item Results may not be transferable
      \end{itemize}
    \item Alternative: Use a real building!
  \end{itemize}
  
  \begin{itemize}
    \item We can't provide access to a building
    \item We can provide access to a simulator
  \end{itemize}
  
\end{frame}

%%%%%%%%%%%%%%%%%%%%%%%%%%%%%%%%%%%%%%%%%%%%%%%%%%%%%%%%%%%%%%%
%%%%%%%%%%%%%%%%%%%%%%%%%%%%%%%%%%%%%%%%%%%%%%%%%%%% Simulation
\begin{frame}
  \frametitle{Simulation}
  
  \vspace{1.4cm}
  \begin{center}
    \begin{tabular}{ccc}
      \scalebox{0.65}{\includeSVG{buildinglogic}} & \hspace{0mm} & \pause \scalebox{0.65}{\includeSVG{simulatorlogic}} \\
      Deployment & & Simulation
    \end{tabular}
  \end{center}
  
\end{frame}

%%%%%%%%%%%%%%%%%%%%%%%%%%%%%%%%%%%%%%%%%%%%%%%%%%%%%%%%%%%%%%%
%%%%%%%%%%%%%%%%%%%%%%%%%%%%%%%%%%%%%%%%%%%%%%%%%%%%% Blueprint
\begin{frame}
  \frametitle{Dummy Building}
  
  \vspace{-8mm}\begin{center}
    \scalebox{0.43}{\includeSVG{building0}}
  \end{center}
  
\end{frame}

%%%%%%%%%%%%%%%%%%%%%%%%%%%%%%%%%%%%%%%%%%%%%%%%%%%%%%%%%%%%%%%
%%%%%%%%%%%%%%%%%%%%%%%%%%%%%%%%%%%%%%%%%%%%%%%%%%%%%%%% System
\begin{frame}
  \frametitle{System}
  
  \vspace{-7mm}\begin{center}
    \scalebox{1.0}{\includeSVG{mediumleveloverview}}
  \end{center}
  
\end{frame}

%%%%%%%%%%%%%%%%%%%%%%%%%%%%%%%%%%%%%%%%%%%%%%%%%%%%%%%%%%%%%%%
%%%%%%%%%%%%%%%%%%%%%%%%%%%%%%%%%%%%%%%%%%%%%%%%%%%%% Blueprint
\begin{frame}
  \frametitle{Blueprint}
  
  \vspace{8mm}\begin{center}
    \scalebox{0.85}{\includeSVG{simplebuilding}}
  \end{center}
  
\end{frame}

%%%%%%%%%%%%%%%%%%%%%%%%%%%%%%%%%%%%%%%%%%%%%%%%%%%%%%%%%%%%%%%
%%%%%%%%%%%%%%%%%%%%%%%%%%%%%%%%%%%%%%%%%%%%%%%%%%%%% Intuition
\begin{frame}
  \frametitle{Intuition}
  
  \vspace{-8mm}\begin{center}
    \scalebox{0.27}{\includeSVG{layers}}
  \end{center}
  
\end{frame}

%%%%%%%%%%%%%%%%%%%%%%%%%%%%%%%%%%%%%%%%%%%%%%%%%%%%%%%%%%%%%%%
%%%%%%%%%%%%%%%%%%%%%%%%%%%%%%%%%%%%%%%%%%%%%%%%%%% Hydro Layer
\begin{frame}
  \frametitle{Hydro Layer}
  
  \vspace{18mm}\begin{center}
    \scalebox{0.4}{\includeSVG{defaulthydro}}
  \end{center}
  
\end{frame}

%%%%%%%%%%%%%%%%%%%%%%%%%%%%%%%%%%%%%%%%%%%%%%%%%%%%%%%%%%%%%%%
%%%%%%%%%%%%%%%%%%%%%%%%%%%%%%%%%%%%%%%%%%%%%%%%%% Energy Layer
\begin{frame}
  \frametitle{Energy Layer}
  
  \vspace{3mm}\begin{center}
    \scalebox{0.4}{\includeSVG{defaultenergy}}
  \end{center}
  
\end{frame}

%%%%%%%%%%%%%%%%%%%%%%%%%%%%%%%%%%%%%%%%%%%%%%%%%%%%%%%%%%%%%%%
%%%%%%%%%%%%%%%%%%%%%%%%%%%%%%%%%%%%%%%%%%%%%%%%% Thermal Layer
\begin{frame}
  \frametitle{Thermal Layer}
  
  \begin{itemize}
    \item Room-level granularity
    \item Internal model: A room connectivity graph
      \begin{itemize}
        \item A heater is represented as a room
        \item An airconditioner is represented as a room
      \end{itemize}
    \item Simulation type: Successive over-relaxation
      \begin{itemize}
        \item Energy is transferred through walls
        \item Rooms affect their neighbors directly
        \item Rooms affect the rest of the building indirectly
      \end{itemize}
  \end{itemize}
  
\end{frame}

%%%%%%%%%%%%%%%%%%%%%%%%%%%%%%%%%%%%%%%%%%%%%%%%%%%%%%%%%%%%%%%
%%%%%%%%%%%%%%%%%%%%%%%%%%%%%%%%%%%%%%%%%%%%%%%%%%% Light Layer
\begin{frame}
  \frametitle{Light Layer}
  
  It's really very simple:
  \begin{center}
  \scalebox{0.8}{
  $
  L =
      \sum\limits_{i=0}^{\mathrm{room}} \mathbf{lamps}[i] +
      \sum\limits_{j=0}^{\mathrm{room}}
      \left(
        \mathbf{has\_blinds}[j] ~?~ \mathbf{blinds}[j] : 1
      \right) \cdot \mathbf{windows}[j] \cdot \mathbf{cloud} \cdot \mathbf{sun}$
  }
  \end{center}
  \begin{itemize}
    \pause
    \item The notation does not follow strict mathematical tradition
    \pause
    \item The essence is that the light in a room is the sum of
    \begin{enumerate}
      \item The light generated by the lamps in the room
      \item The light being let through the windows in the room
    \end{enumerate}
  \end{itemize}
  
\end{frame}

%%%%%%%%%%%%%%%%%%%%%%%%%%%%%%%%%%%%%%%%%%%%%%%%%%%%%%%%%%%%%%%
%%%%%%%%%%%%%%%%%%%%%%%%%%%%%%%%%%%%%%%%%%%%%%%%%%%%% Interface
\begin{frame}
  \frametitle{How to communicate with the simulator}
  
  \begin{itemize}
    \item RESTless web interface
    \item 4 type of queries:
    \begin{itemize}
          \item Buildings available
          \item Building blueprint
          \item Get a service value
          \item Set a service value
    \end{itemize}
    \item Each service has a unique id (e.g. a specific light)
    \item You find the service id under the building blueprint
    \item How to get started? 

 \end{itemize}
  
\end{frame}

%%%%%%%%%%%%%%%%%%%%%%%%%%%%%%%%%%%%%%%%%%%%%%%%%%%%%%%%%%%%%%%
%%%%%%%%%%%%%%%%%%%%%%%%%%%%%%%%%%%%%%%%%%%%%%%%%%%% Conclusion
\begin{frame}
  \frametitle{More}
  
  \begin{itemize}
    \item A live building simulator is located at\\
            \scalebox{.8}{\url{http://gsd.itu.dk/api/user/}} 
    \item If you wish, you can get the source and run your own building simulator. For example:
      \begin{itemize}
        \item Work completely independently of other users
        \item Run simulation of the building behavior using your own timings (currently the building updates all sensors every 15 seconds)
      \end{itemize}
    \item Links:
    \begin{itemize}
      \item Tastypie documentation: \\
            \scalebox{.8}{\url{http://django-tastypie.readthedocs.org/en/latest/interacting.html}}
             
      \item Building documentation: \\
            \scalebox{.8}{\url{https://gsd.wikit.itu.dk/GSD2013+Building+Simulator}}
      \item Source code:  \\
            \scalebox{.8}{\url{https://github.com/javigon/bisp/}}
    \end{itemize}
    \item Let us know how the simulator can be improved (or send patches)!
  \end{itemize}
  
\end{frame}

\end{document}
